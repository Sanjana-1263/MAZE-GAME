\documentclass[14pt]{beamer}
\title{Maze}
\date{27-03-2021}
\author[Bvrith]{Harini: 19WH1A0510: CSE \\  Nidhi: 19WH1A0507: CSE \\ Mounika: 20WH5A0513: CSE \\ Preethi: 19WH1A0215: EEE  \\ Sanjana: 19WH1A1263: IT}
\usepackage{xcolor}
\definecolor{blendedblue}{rgb}{0.2,0.2,0.7}
\institute{\normalsize{\color{blendedblue}{BVRIT HYDERABAD College of Engineering for Women}}}
\usefonttheme{serif}
\usepackage{bookman}
\usepackage{hyperref}
\usepackage[T1]{fontenc}
\usepackage{graphicx}
\usecolortheme{orchid}
\beamertemplateballitem
\graphicspath{./home/user/Desktop/maze}
\graphicspath{./home/user/Desktop/maze}
\graphicspath{./home/user/Desktop/maze}
\graphicspath{./home/user/Desktop/maze}
\graphicspath{./home/user/Desktop/maze}
\graphicspath{./home/user/Desktop/maze}

\begin{document}
    \begin{frame}
        \titlepage
    \end{frame}
    \begin{frame}
	\frametitle{Problem Statement}
        \begin{itemize}
	    \item Consider a person placed at (0, 0) in a square matrix of order N*N. It has to reach the destination at (n-1, n-1)
	    \item Find a possible path that the person can take to reach from source to destination
	    \item Directions in which the person can move are ‘U'(up), ‘D'(down), ‘L’ (left), ‘R’ (right)
	\end{itemize}
    \end{frame}
    \begin{frame}
	    \frametitle{Problem Statement}
	    \begin{figure}[htp]
                        \centering
                         \includegraphics[width=10cm]{maze1.png}
			 \caption{Maze Example}
                 \end{figure}
    \end{frame}
    \begin{frame}
	\frametitle{Approach}
	\begin{itemize}
		\item We first start from the source and move in a direction where the path is not blocked (not 0)
		\item If taken path makes us reach the destination then the problem is solved else, we backtrack
	\end{itemize}
    \end{frame}
    \begin{frame}
	\frametitle{Code}
	    \begin{figure}[htp]
                        \centering
                         \includegraphics[width=10cm]{code1.png}
                 \end{figure}
    \end{frame}
    \begin{frame}
	\frametitle{Code}
	    \begin{figure}[htp]
		    \centering
		    \begin{minipage}[b]{0.45\textwidth}
			    \includegraphics[width=\textwidth]{code2.png}
		    \end{minipage}
		    \hfill
		    \begin{minipage}[b]{0.45\textwidth}
			    \includegraphics[width=\textwidth]{code3.png}
		    \end{minipage}
	    \end{figure}
    \end{frame}
    \begin{frame}
	\frametitle{Statistics}
	\begin{itemize}
		\item Lines of Code - 63
		\item Number of functions - 4
	\end{itemize}
    \end{frame}
    \begin{frame}
	\frametitle{Tools}
	\begin{itemize}
		\item OS - Ubuntu, Windows
		\item Programming language (version Python 3.8.1)
		\item GitLab
	\end{itemize}
    \end{frame}
    \begin{frame}
        \frametitle{Learnings}
	\begin{itemize}
	    \item Backtracking 
	    \item Inserting Images in Latex
	    \item Inserting Hyperlinks in latex
	    \item GIT - push and pull
	\end{itemize}
    \end{frame}
    \begin{frame}
	\frametitle{Challenges}
        \begin{itemize}
	    \item Matrix as an input from user
	    \item Implementing Recursion
	    \item GIT - pushing files to gitlab
        \end{itemize}
    \end{frame}
    \begin{frame}
	\frametitle{GIT Repo}
	\begin{itemize}
		\begin{figure}[htp]
			\centering
			\includegraphics[width=10cm][repo.jpeg]
		\end{figure}
	\end{itemize}
    \end{frame}
    \begin{frame}
	\frametitle{Demo}
	     \begin{figure}[htp]
                        \centering
                         \includegraphics[width=10cm]{output.png}
          \end{figure}	
    \end{frame}
    \begin{frame}
	\begin{center}
	     THANK YOU
	\end{center}
    \end{frame}
\end{document}

